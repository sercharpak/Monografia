\documentclass{article}
% Use LaTex, standard version with no macros (i.e. do not modify the header)
% PLEASE READ CAREFULLY ALL THE INSTRUCTIONS for preparing this ABSTRACT
%
% These instructions facilitate the organization of conferences.
%
% It will be used for the reviewing process and the booklet of the conference
% 
% 

\begin{document}

\title{Detection of galaxies superclusters in simulated cosmological structures}
\author{Sergio Daniel Hernandez Charpak\\
{\small  Universidad de los Andes, Cra 1 Nº 18A- 12 Bogotá, (Colombia)} \\
{\small {\itshape Email} : sd.hernandez204@uniandes.edu.co}}



% Please indicate whether you submit your contribution to the plenary session or to a conference, or to a poster session, and the Topic ID
% The Topic identification numbers are the following  
% 1.		Distances, H0, and peculiar velocities
% 2.		Redshift surveys and implied velocity fields
% 3.		Linear and non-linear velocity field reconstructions
% 4.		Comparison/Reconciliation: velocities ? densities ? observed galaxie
% 5.		Bulk flows
% 6.		Cosmological parameters
% 7.		Initial conditions & constrained simulations
% 8.		Others
%
% e.g., Plenary 6

\date{Contribution type : Plenary 4}

% It is recommended to write a concise and understandable abstract not exceeding ten (10) lines of text and a maximum of three (3) references.
% Please verify that LaTeX compiles your file without errors
%
% Use LATEX conventions for equations, mathematical symbols etc..
% Do not separate your abstract into chapters, sections etc; the abstract should be one unit (but it may of course contain several paragraphs).
% Do not include figures in the abstract.

\maketitle


\begin{abstract} 
Galaxies form larger-scale structures known as clusters and superclusters. Tully et al
used the V-Web algorithm on the flow map of the local universe to define our home
supercluster of galaxies, which they names Laniakea, infinite heaven in Hawaiian.\\
In this work we used similar techniques to detect galaxies superclusters in different
N-Body simulations. Using region growing algorithms, an image segmentation method, we
detect several large-scale structures of the same order of magnitude as Laniakea (100 Mpc/h).
We then perform comparations of different properties of these structures with our home supercluster, Laniakea.
\end{abstract}


% Please keep numbers for the labels of references, as described above
\begin{thebibliography}{00} 
\bibitem{1} R. Brent Tully, Hlne. Courtois, Yehuda Hoffman and Daniel Pomarde. 
{\em The Laniakea Supercluster of galaxies}, Nature, 513 (7516):71-73, September 2014 


\end{thebibliography} 



\end{document} 