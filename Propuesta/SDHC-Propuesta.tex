\documentclass[12pt]{article}

%PACKAGES
\usepackage{graphicx}
\usepackage{epstopdf}
\usepackage[english]{babel}

\usepackage{hyperref}
\usepackage[left=3cm,top=3cm,right=3cm,nohead,nofoot]{geometry}
\usepackage{braket}
\usepackage{datenumber}
\usepackage{placeins}


%COMMANDS

\begin{document}

\begin{center}
\Huge
Detecci\'on de superc\'{u}mulos de galaxias en estructuras
cosmol\'{o}gicas simuladas\\  
\vspace{3mm}
\Large Sergio Daniel Hern\'{a}ndez Charpak

\large
200922618

\vspace{2mm}
\Large
Director: Jaime E. Forero-Romero

\normalsize
\vspace{2mm}

\today
\end{center}


\normalsize
\section{Introducci\'{o}n}
%Introducción a la propuesta de Monografía. Debe incluir un breve resumen del estado del arte del problema a tratar. También deben aparecer citadas todas las referencias de la bibliografía (a menos de que se citen más adelante, en los objetivos o metodología, por ejemplo)
En el Universo visto a grandes escalas las galaxias se agrupan en
estructuras amplias semejantes a una red con filamentos que atraviesan
grandes regiones vac\'ias y se cruzan en regiones denominadas superc\'{u}mulos.
Aunque estas estructuras son f\'aciles de detectar a simple vista, hay
diferentes posibilidades para delimitarlas a partir de criterios
f\'isicos \cite{gott_iii_map_2005}.    
\\

Una propuesta para definir un superc\'umulo es utilizar el flujo de
velocidades de las galaxias en esa regi\'on del espacio.
En un superc\'umulo las galaxias tienden a fluir hacia la regi\'on
m\'as densa debido al proceso de atracci\'on gravitacional.
De esta manera, las regiones espaciales donde el flujo de galaxias es
convergente  representar\'ian superc\'umulos diferentes.\\
\\
Recientemente un equipo de astr\'{o}nomos construy\'o un mapa del flujo
de velocidades de galaxias en el Universo local en una escala de
cientos de millones de a\~nos luz\cite{tully_cosmicflows-2_2013}. 
En dicho mapa se encuentran puntos en donde este flujo convege. 
Con base en este mapa este equipo identific\'{o} a Laniakea, el
superc\'umulo de galaxias que incluye a nuestra galaxia, la V\'ia
L\'{a}ctea\cite{tully_laniakea_2014}. 
\\   

En esta monograf\'{i}a nos proponemos desarrollar un m\'etodo para
detectar cientos de superc\'{u}mulos de galaxias en simulaciones
cosmol\'ogicas.  
Con esto buscamos cuantificar si Laniakea puede considerarse como una
estructura at\'ipica en el Universo.
\\

\section{Objetivo General}
%Objetivo general del trabajo. Empieza con un verbo en infinitivo.

Detectar y caracterizar superc\'{u}mulos de galaxias en simulaciones
cosmol\'{o}gicas. 
\\

\section{Objetivos Espec\'{i}ficos}
%Objetivos específicos del trabajo. Empiezan con un verbo en infinitivo.

\begin{itemize}
	\item Simular la distribuci\'on de materia en escalas
          cosmol\'{o}gicas. 
	\item Desarrollar m\'{e}todos de detecci\'on de
          superc\'{u}mulos de galaxias en simulaciones.
	\item Caracterizar los superc\'umulos encontrados en la
          simulaci\'on.
        \item Comparar las caracter\'isticas de los conjuntos de
          superc\'umulos detectados en          la simulaci\'on con
          Laniakea. 
\end{itemize}

\section{Metodolog\'{i}a}
%Exponer DETALLADAMENTE la metodología que se usará en la Monografía. 
%Monografía computacional: ¿Cómo se harán las simulaciones? ¿Qué requerimientos computacionales se necesitan? ¿Qué espacios físicos o virtuales se van a utilizar?

Esta monograf\'{i}a utiliza en su mayor\'{i}a m\'etodos computacional.  
Usaremos los lenguajes C y Python, ya que ambos son de licencia
libre. 
Para simulaciones de la distribuci\'on de materia en escalas
cosmol\'ogicas utilizaremos en el cluster de la Universidad utilizando
c\'odigos de disponibilidad p\'ublica.
En el mismo cluster implementaremos el
algoritmo que detecta los superc\'umulos.
\\

Como estudiante debo entender como se realizan las
simulaciones. Deber\'{e} familiarizarme con los datos manejados en
este entorno y poderlos manipular sin gran dificultad.
\\

Una vez generada una simulaci\'on cosmol\'{o}gica, 
deber\'{e} estudiarla en m\'{a}s detalle para familiarizarme con
ella.
\\
Deber\'{e} familiarizarme con el trabajo hecho por Hoffman et al\cite{hoffman_kinematic_2012} en el algoritmo V-web, utilizado en este entorno. Identificar\'{e} el campo de velocidades de la materia. Para ello debo tener un entendimiento de la relaci\'{o}n entre campo de velocidades  y distribuci\'{o}n de materia. 

Luego probar\'{e} los m\'{e}todos previamente estudiados para obtener unos
primeros resultados.  
\\

Deber\'{e} refinar estos m\'{e}todos con el fin de diferenciar
superc\'{u}mulos en las estructuras cosmol\'{o}gicas. 
\\

Finalmente caracterizar\'{e} los superc\'umulos para compararlos con las
propiedades del superc\'umulo Laniakea.

\section{Cronograma}

\begin{table}[htb]
	\begin{tabular}{|c|cccccccccccccccc| }
	\hline
	Tareas $\backslash$ Semanas & 1 & 2 & 3 & 4 & 5 & 6 & 7 & 8 & 9 & 10 & 11 & 12 & 13 & 14 & 15 & 16  \\
	\hline
	1 & X & X & X &   &   &   &   &   &   &   &   &   &   &   &   &   \\
	2 &   & X & X & X & X &   &   &   &   &   &   &   &   &   &   &   \\
	3 &   & X & X & X & X &   &   &   &   &   &   &   &   &   &   &   \\
	4 &   &   & X & X & X & X &   &   &   &   &   &   &   &   &   &   \\
	5 &   &   &   &   & X & X & X & X &  &  &   &   &   &   &   &   \\
	6 &   &   &   &   &   &   &  & X &  &  &   &   &   &   &   &   \\
	7 &   &   &   &   &   &   & X & X & X & X & & & &  & & \\  
	8 &   &   &   &   &   &   & X & X & X & X &  &  & & & &  \\
    9 &   &   &   &   &   &   &  &  &  &  & X & X &   &   &   &   \\
	10 &   &   &   &   &   &   &  &  &  &  & X & X  & X  & X  & X  & X \\
	\hline
	\end{tabular}
\end{table}
\vspace{1mm}
\FloatBarrier

\begin{itemize}
	\item Tarea 1: Revisi\'{o}n de la literatura. 
	\item Tarea 2: Realizaci\'on de simulaciones de estructuras
          cosmol\'{o}gicas. 
	\item Tarea 3: Revisi\'{o}n de m\'{e}todos de
          diferenciaci\'{o}n de superc\'{u}mulos 
	\item Tarea 4: Desarrollo del c\'{o}digo para detectar 
          superc\'{u}mulo de galaxias en simulaciones, ensayos con 
          con simulaciones de prueba.
    \item Tarea 5: Realizaci\'{o}n del documento previo que debe presentarse en el 30\%
    \item Tarea 6: Sustentaci\'{o}n oral del documento previo
	\item Tarea 7: Ejecutar el c\'{o}digo y verificar los
          resultados sobre las simulaciones hechas en la Tarea 2.
	\item Tarea 8: Caracterizar las propiedades de los
          superc\'umulos en la simulaci\'on.
	\item Tarea 9: Comparar con los datos observacionales de
          Laniakea. 
	\item Tarea 10: Escribir el documento final.
\end{itemize}

\section{Expertos}
%Nombres de por lo menos 3 profesores que conozcan del tema. Uno de ellos debe ser profesor de planta de la Universidad de los Andes.

\begin{itemize}
	\item Marek Nowakowski (Uniandes)
	\item Nelson Padilla (PUC, Chile)
	\item Juan Carlos Mu\~noz Cuartas (UdeA, Colombia)
\end{itemize}

\bibliographystyle{unsrt}
\renewcommand\refname{Referencias}
\bibliography{references}

\section*{Firma del Director}
\vspace{2.5cm}

Jaime E. Forero-Romero


\end{document} 
