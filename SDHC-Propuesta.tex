\documentclass[12pt]{article}

%PACKAGES
\usepackage{graphicx}
\usepackage{epstopdf}
\usepackage[english]{babel}
\usepackage[latin5]{inputenc}
\usepackage{hyperref}
\usepackage[left=3cm,top=3cm,right=3cm,nohead,nofoot]{geometry}
\usepackage{braket}
\usepackage{datenumber}

%COMMANDS

\begin{document}

\begin{center}
\Huge
Differenciaci\'{o}n de Superc\'{u}mulos de Galaxias en Estructuras Cosmol\'{o}gicas Simuladas\\
\vspace{3mm}
\Large Sergio Daniel Hern\'{a}ndez Charpak

\large
200922618

\vspace{2mm}
\Large
Advisor: Jaime E. Forero-Romero

\normalsize
\vspace{2mm}

\today
\end{center}


\normalsize
\section{Introducci\'{o}n}
%Introducción a la propuesta de Monografía. Debe incluir un breve resumen del estado del arte del problema a tratar. También deben aparecer citadas todas las referencias de la bibliografía (a menos de que se citen más adelante, en los objetivos o metodología, por ejemplo)

Un equipo de cient\'{i}ficos liderado por Brent Tully\cite{NatureTullyCourtois}, de la Universidad de Hawai\'o en Honolulu encontr\'{o} un m\'{e}todo para diferenciar un superc\'{u}mulo de galaxias de otro. Estudian los efectos de las galaxias en vez de de buscar la materia misma. Haciendo un mapa del flujo de velocidades de galaxias alrededor nuestro, identificaron el supercluster que incluye a nuestra galaxia, la Vida L\'{a}ctea, y lo nombraron Laniakea.
\\


Los m\'{e}todos usados durante este estudio son:
%TODO Acabar de leer los papers.
\\

En esta monograf\'{i}a buscamos un entendimiento de los m\'{e}todos usados para la diferenciaci\'{o}n de superc\'{u}mulos de galaxias. Se simular\'{a}n entonces estructuras cosmol\'{o}gicas en las que se aplicaran los m\'{e}todos de diferenciaci\'{o}n. Al obtener una agilidad en los m\'{e}todos se podr\'{a}n simular superc\'{u}mulos de propiedas similares a Lanikea.
\\

\section{Objetivo General}
%Objetivo general del trabajo. Empieza con un verbo en infinitivo.

Diferenciar superc\'{u}mulos de galaxias en diferentes estructuras cosmol\'{o}gicas para as\'{i} entender los m\'{e}todos actuales.
\\

\section{Objetivos Espec\'{i}ficos}
%Objetivos específicos del trabajo. Empiezan con un verbo en infinitivo.

\begin{itemize}
	\item Simular Estructuras Cosmol\'{o}gicas.
	\item Describir los m\'{e}todos de diferenciaci\'{o}n de superc\'{u}mulos de galaxias.
	\item Aplicar estos m\'{e}todos en las estrucutras Cosmol\'{o}gicas
\end{itemize}

\section{Metodolog\'{i}a}
%Exponer DETALLADAMENTE la metodología que se usará en la Monografía. 
%Monografía computacional: ¿Cómo se harán las simulaciones? ¿Qué requerimientos computacionales se necesitan? ¿Qué espacios físicos o virtuales se van a utilizar?

Esta monograf\'{i}a es en su mayor\'{i}a computacional. Se usar\'{a}n los lenguajes C y Python, ya que ambos son de licencia libre. Las simulaciones ser realizar\'{a}n en el computador del estudiante.  
\\

El estudiante debe entender como se realizan las simulaciones. Deber\'{a} familiarizarse con los datos manejados en este entorno. 
\\

Una vez generada una estructura cosmol\'{o}gica, el estudiante deber\'{a} estudiarla en m\'{a}s detalle para familiarizarse con ella. Identificar\'{a} el campo de velocidades peculiares. Luego probar\'{a} los m\'{e}todos previamente estudiados para obtener unos primeros resultados.
\\

El estudiante refina sus m\'{e}todos con el fin de diferenciar superc\'{u}mulos en las estructuras cosmol\'{o}gicas.
\\

\section{Schedule}

\begin{table}[htb]
	\begin{tabular}{|c|cccccccccccccccc| }
	\hline
	Tasks $\backslash$ Weeks & 1 & 2 & 3 & 4 & 5 & 6 & 7 & 8 & 9 & 10 & 11 & 12 & 13 & 14 & 15 & 16  \\
	\hline
	1 & X & X & X &   &   &   &   &   &   &   &   &   &   &   &   &   \\
	2 &   & X & X &   &   &   &   &   &   &   &   &   &   &   &   &   \\
	3 &   & X & X & X &   &   &   &   &   &   &   &   &   &   &   &   \\
	4 &   &   &   & X & X & X &   &   &   &   &   &   &   &   &   &   \\
	5 &   &   &   &   &   & X &   &   &   &   &   &   &   &   &   &   \\
	6 &   &   &   &   &   &   & X & X & X &   &   &   &   &   &   &   \\
	7 &   &   &   &   &   &   &   &   & X &   &   &   &   &   &   &   \\
	8 &   &   &   &   &   &   &   &   &   & X & X &   &   &   &   &   \\
	9 &   &   &   &   &   &   &   &   &   &   & X & X &   &   &   &   \\
	10 &  &   &   &   &   &   &   &   &   &   &   & X & X &   &   &   \\
	11 &  &   &   &   &   &   &   &   &   &   &   &   &   & X & X &   \\
	12 &  &   &   &   &   &   &   &   & X & X & X & X & X & X & X & X \\
	\hline
	\end{tabular}
\end{table}
\vspace{1mm}

\begin{itemize}
	\item Task 1: Revisi\'{o}n de la literatura. 
	\item Task 2: Revisi\'{o}n de simulaciones de estructuras cosmol\'{o}gicas.
	\item Task 3: Revisi\'{o}n de m\'{e}todos de diferenciaci\'{o}n de superc\'{u}mulos de galaxias.
	\item Task 4: Escribir el c\'{o}digo de la simulaci\'{o}n de la estruct\'{u}ra cosmol\'{o}gica. 
	\item Task 5: Ejecutar el c\'{o}digo y observar los resultados.
	\item Task 6: Escribir el c\'{o}digo para diferenciar un superc\'{u}mulo de galaxias. 
	\item Task 7: Ejecutar el c\'{o}digo y observar los resultados.
	\item Task 8: Escribir el c\'{o}digo para diferenciar varios superc\'{u}mulos de galaxias. 
	\item Task 9: Ejecutar el c\'{o}digo y observar los resultados.
	\item Task 10: Observar resultados con diversas estructuras cosmol\'{o}gicas.
	\item Task 11: Comparar con datos observacionales.
	\item Task 12: Escribir el documento.
\end{itemize}

\section{Expertos}
%Nombres de por lo menos 3 profesores que conozcan del tema. Uno de ellos debe ser profesor de planta de la Universidad de los Andes.

\begin{itemize}
	\item 
	\item 
	\item  
\end{itemize}

\bibliographystyle{unsrt}
%\bibliographystyle{amsplain}
\bibliography{references}

\section*{Firma del Director}
\vspace{2.5cm}

Jaime E. Forero-Romero


\end{document} 