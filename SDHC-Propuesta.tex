\documentclass[12pt]{article}

%PACKAGES
\usepackage{graphicx}
\usepackage{epstopdf}
\usepackage[english]{babel}
\usepackage[latin5]{inputenc}
\usepackage{hyperref}
\usepackage[left=3cm,top=3cm,right=3cm,nohead,nofoot]{geometry}
\usepackage{braket}
\usepackage{datenumber}

%COMMANDS

\begin{document}

\begin{center}
\Huge
Detecci\'on de superc\'{u}mulos de galaxias en estructuras
cosmol\'{o}gicas simuladas\\  
\vspace{3mm}
\Large Sergio Daniel Hern\'{a}ndez Charpak

\large
200922618

\vspace{2mm}
\Large
Advisor: Jaime E. Forero-Romero

\normalsize
\vspace{2mm}

\today
\end{center}


\normalsize
\section{Introducci\'{o}n}
%Introducción a la propuesta de Monografía. Debe incluir un breve resumen del estado del arte del problema a tratar. También deben aparecer citadas todas las referencias de la bibliografía (a menos de que se citen más adelante, en los objetivos o metodología, por ejemplo)

Recientemente un equipo de astr\'onomos construy\'o un mapa del flujo
de velocidades de galaxias en el Universo local en una escala de
cientos de millones de a\~nos luz.
A partir de este mapa identificaron a Laniakea, el superc\'umulo de galaxias que
incluye a nuestra galaxia, la Vida L\'{a}ctea
\\  


Los m\'{e}todos usados durante este estudio son:
%TODO Acabar de leer los papers.
\\

En esta monograf\'{i}a nos proponemos desarrollar un m\'etodo para
detecatr cientos de superc\'{u}mulos de galaxias en simulaciones. 
Con esto buscamos cuantificar si Laniakea puede considerarse como una
estructura at\'ipica en el Universo.
\\

\section{Objetivo General}
%Objetivo general del trabajo. Empieza con un verbo en infinitivo.

Detectar y caracterizar superc\'{u}mulos de galaxias en simulaciones
cosmol\'{o}gicas. 
\\

\section{Objetivos Espec\'{i}ficos}
%Objetivos específicos del trabajo. Empiezan con un verbo en infinitivo.

\begin{itemize}
	\item Simular la distribuci\'on de materia en escalas
          cosmol\'{o}gicas. 
	\item Desarrollar m\'{e}todos de detecci\'on de
          superc\'{u}mulos de galaxias en simulaciones.
	\item Caracterizar los superc\'umulos encontrados en la
          simulaci\'on.
        \item Comparar las caracter\'isticas de Laniakea con las
          propiedades de los conjuntos de superc\'umulos detectados en
          la simulaci\'on.
\end{itemize}

\section{Metodolog\'{i}a}
%Exponer DETALLADAMENTE la metodología que se usará en la Monografía. 
%Monografía computacional: ¿Cómo se harán las simulaciones? ¿Qué requerimientos computacionales se necesitan? ¿Qué espacios físicos o virtuales se van a utilizar?

Esta monograf\'{i}a es en su mayor\'{i}a computacional. 
Se usar\'{a}n los lenguajes C y Python, ya que ambos son de licencia
libre. 
Las simulaciones de la distribuci\'on de materia en escalas
cosmol\'ogicas se har\'an en el cluster de la Universidad utilizando
c\'odigos de disponibilidad p\'ublica.
En el mismo cluster se desarrollar\'a la implementaci\'on del
algoritmo que detecta los superc\'umulos.
\\

El estudiante debe entender como se realizan las
simulaciones. Deber\'{a} familiarizarse con los datos manejados en
este entorno.  
\\

Una vez generada una simulaci\'on cosmol\'{o}gica, el estudiante
deber\'{a} estudiarla en m\'{a}s detalle para familiarizarse con
ella. Identificar\'{a} el campo de velocidades de la materia. 
Luego probar\'{a} los m\'{e}todos previamente estudiados para obtener unos
primeros resultados.  
\\

El estudiante refina sus m\'{e}todos con el fin de diferenciar
superc\'{u}mulos en las estructuras cosmol\'{o}gicas. 
\\

El estudiante caracteriza los superc\'umulos para compararlos con las
propiedades del superc\'umulo Laniakea.

\section{Schedule}

\begin{table}[htb]
	\begin{tabular}{|c|cccccccccccccccc| }
	\hline
	Tasks $\backslash$ Weeks & 1 & 2 & 3 & 4 & 5 & 6 & 7 & 8 & 9 & 10 & 11 & 12 & 13 & 14 & 15 & 16  \\
	\hline
	1 & X & X & X &   &   &   &   &   &   &   &   &   &   &   &   &   \\
	2 &   & X & X & X & X &   &   &   &   &   &   &   &   &   &   &   \\
	3 &   & X & X & X & X &   &   &   &   &   &   &   &   &   &   &   \\
	4 &   &   & X & X & X & X &   &   &   &   &   &   &   &   &   &   \\
	5 &   &   &   &   &   &   & X & X & X & X &   &   &   &   &   &   \\
	6 &   &   &   &   &   &   & X & X & X & X &   &   &   &   &   &   \\
	7 &   &   &   &   &   &   &   &   &   &   & X & X &   &   &   &   \\
	8 &   &   &   &   &   &   &   &   &   &   & X & X & X & X & X & X \\
	\hline
	\end{tabular}
\end{table}
\vspace{1mm}

\begin{itemize}
	\item Task 1: Revisi\'{o}n de la literatura. 
	\item Task 2: Realizaci\'on de simulaciones de estructuras
          cosmol\'{o}gicas. 
	\item Task 3: Revisi\'{o}n de m\'{e}todos de
          diferenciaci\'{o}n de superc\'{u}mulos 
	\item Task 4: Desarrollo del c\'{o}digo para detectar 
          superc\'{u}mulo de galaxias en simulaciones, ensayos con 
          con simulaciones de prueba.
	\item Task 5: Ejecutar el c\'{o}digo y verificar los
          resultados sobre las simulaciones hechas en la Task 2.
	\item Task 6: Caracterizar las propiedades de los
          superc\'umulos en la simulaci\'on.
	\item Task 7: Comparar con los datos observacionales de
          Laniakea. 
	\item Task 8: Escribir el documento.
\end{itemize}

\section{Expertos}
%Nombres de por lo menos 3 profesores que conozcan del tema. Uno de ellos debe ser profesor de planta de la Universidad de los Andes.

\begin{itemize}
	\item Marek Nowakowski (Uniandes)
	\item Nelson Padilla (PUC, Chile)
	\item Juan Carlos Mu\~noz Cuartas (UdeA, Colombia)
\end{itemize}

\bibliographystyle{unsrt}
%\bibliographystyle{amsplain}
\bibliography{references}

\section*{Firma del Director}
\vspace{2.5cm}

Jaime E. Forero-Romero


\end{document} 
